\documentclass[%
 aip,
% jmp,
% bmf,
% sd,
% rsi,
cp,  % Conference Proceedings
 amsmath,amssymb,%nobibnotes,
 preprint,%
% reprint,%
%author-year,%
%author-numerical,%
]{revtex4-2}

\usepackage{graphicx}% Include figure files
\usepackage{dcolumn}% Align table columns on decimal point
\usepackage{bm}% bold math
%\usepackage[mathlines]{lineno}% Enable numbering of text and display math
%\linenumbers\relax % Commence numbering lines

\usepackage[utf8]{inputenc}
\usepackage[T1]{fontenc}
%% Loads a Times-like font. You can also load
%% {newtxtext,newtxtmath}, but not {times}, 
%% {txfonts} nor {mathtpm} as these packages
%% are obsolete and have been known to cause problems.
\usepackage{mathptmx} 

\newcommand{\vecb}[3]{
\bm{\vec{#1}_{#2}^{#3}}
}

\newcommand{\uvecb}[3]{
\bm{\hat{#1}_{#2}^{#3}}
}

\begin{document}

\title{Monte Carlo Simulation of X-ray Transport}% Force line breaks with \\

\author{John Meneghini} % Write as First name Surname
 \email[Corresponding author: ]{john.meneghini@stvincent.edu}
\affiliation{
 Department of Physics, Saint Vincent College, Latrobe, PA 15650
}

\date{\today} % It is always \today, today, but any date may be explicitly specified
              % Not printed for conference proceedings

\begin{abstract}
An article usually includes an abstract, a concise summary of the work
covered at length in the main body of the article. It is used for
secondary publications and for information retrieval purposes.
\end{abstract}

\maketitle

\section{INTRODUCTION}



\section{TRANSPORT THEORY}
\par In order to represent the computational domain discretely, space is broken up into a grid of voxels (a pixel with volume), with each voxel being assigned a particular material ID depending on the geometries and compounds/elements in the domain. With our discrete space, given a photon position $\vecb{r}{}{}$, the corresponding voxel in which the photon resides can be calculated. Therefore, a particular material can be associated with all possible $\vecb{r}{}{}$'s in the domain.

\par A photon's position in space after taking the $n$-th step in the domain $\vecb{r}{n}{}$ is represented by the following parametric ray equation:
\begin{equation}
    \vecb{r}{n}{} = \vecb{r}{n-1}{} + \uvecb{d}{}{} t_n,
\end{equation}
where $\vecb{r}{n-1}{}$ is the initial position before the $n$-th step, $\uvecb{d}{}{}$ is a unit vector in the direction of the step, and $t_n$ is the distance of the $n$-th step.

\par In order to sample $t_n$, we utilize the following probability distribution function (PDF) $p(t)$ of the distance traveled $t$ by a photon of energy $E$ through material $M$ before interacting:
\begin{equation}
p(t) = n\sigma \exp{\left[-t(n\sigma)\right]},
\end{equation}
where $n$ is the number density of $M$ and $\sigma = \sigma(E, M)$ is the microscopic cross-section of $M$ at $E$.
\par Using the inversion method for sampling a PDF, it follows that random values of the free path $t$ can be generated with the following equation:
\begin{equation}
    t = -\frac{1}{n\sigma} \ln \gamma,
\end{equation}
where $\gamma$ is a uniformly distributed random number in the interval $(0, 1)$.
\\

\subsection{SURFACE AND DELTA-TRACKING}

\par If, after taking a step, the photon lands in a voxel with a different material, then the corresponding free path for the new material to be accounted for. This method, called surface-tracking, requires photons to be stopped at voxel boundaries and intersections with surrounding voxels to be calculated, which can be computationally intensive for materials that have a large average free path. 
\par Alternatively, the delta-tracking algorithm offers a solution by sampling the maximum cross-section $\sigma_{\text{max}}$ in the computational domain. This, in turn, brings down the average free path to the minimum in the domain. To account for this decrease in free path, the algorithm introduces delta interactions as an alternative to real interactions, resulting in no change to the energy or direction. The probability of delta interaction $P_{\delta}$ is given by the following equation:

\begin{equation} \label{eq:4}
    P_{\delta} = \frac{\sigma_{\text{max}}(E) - \sigma(E, M)}{\sigma_{\text{max}}(E)},
\end{equation}

where $E$ is the energy of the photon undergoing the step, and $M$ is the material corresponding to the position of the photon at the end of the step. Note that when the photon lands in the material corresponding to the maximum cross-section, $\sigma(E, M) = \sigma_{\text{max}}$ and $P_{\delta} = 0$. On the contrary, if the photon landed in air and the domain's maximum cross-section corresponded to lead, then $\sigma(E, M) << \sigma_{\text{max}}$, making $P_{\sigma} \approx 1$.

\par Overall, delta-tracking is significantly more computationally efficient for domains with similar cross-sections and can be shown to yield equivalent results to surface-tracking.

\subsection{PHOTON INTERACTIONS}
If a delta interaction does not occur, then a real interaction is sampled. Therefore, the probability of a real interaction $P_r$ is directly related to $P_\delta$ by

\begin{equation}
    P_r = 1 - P_\delta,
\end{equation}
where $P_\delta$ is given by Eq. \ref{eq:4}.
\\
\par For x-rays, there are three possible photon interactions: \\

\large{\bf{1. Photoelectric Effect}}

In the following model for the photoelectric effect, 

\section{METHODS}





\begin{acknowledgments}
We wish to acknowledge the support of the author community in using
REV\TeX{}, offering suggestions and encouragement, testing new versions,
\dots.
\end{acknowledgments}

\nocite{*}
\bibliography{aipsamp}% Produces the bibliography via BibTeX.

\end{document}
%
% ****** End of file aipsamp.tex ******


\section{TRANSPORT THEORY}
\par To represent the computational domain discretely, space is broken up into a grid of voxels (a pixel with volume), with each voxel being assigned a particular material ID depending on the geometries and compounds/elements in the domain. Within our discrete space, given a photon position $\vecb{r}{}{}$, the corresponding voxel in which the photon resides can be calculated. Therefore, all possible $\vecb{r}{}{}$'s can be assigned a particular material $M$ in the domain.

\par A photon's position in space after taking the $n$-th step in the domain, $\vecb{r}{n}{}$, is represented by the following parametric ray equation:
\begin{equation} \label{eq:1}
    \vecb{r}{n}{} = \vecb{r}{n-1}{} + \uvecb{d}{}{} t_n,
\end{equation}
where $\vecb{r}{n-1}{}$ is the initial position before the $n$-th step, $\uvecb{d}{}{}$ is a unit vector in the direction of the step, and $t_n$ is the length of the $n$-th step.

\par To randomly sample $t_n$ in a homogenous domain, we utilize the following probability density function (PDF) $p(t)$ of the distance traveled $t$ by a photon of energy $E$ through material $M$ before interacting:
\begin{equation} \label{eq:2}
    p(t) = n\sigma \exp{\left[-t(n\sigma)\right]},
\end{equation}
where $n$ is the number density of $M$ and $\sigma = \sigma(E, M)$ is the microscopic cross-section of $M$ at $E$.
\par Using the inversion method for sampling a PDF on Equation \ref{eq:2}, it follows that random values of the free path $t$ can be generated with the following equation:
\begin{equation} \label{eq:3}
    t = -\frac{1}{n\sigma} \ln \gamma,
\end{equation}
where $\gamma$ is a uniformly distributed random number in the interval $[0, 1]$. This value of $t$ is sampled for each step and is used as $t_n$ in Equation \ref{eq:1} to determine the length of the $n$-th step.
\par For further reading on the theory presented in this section, see \cite{vassiliev_monte_2017}.

\subsection{SURFACE AND $\delta$-TRACKING}

\par If, after taking a step, the photon lands in a voxel with a different material (an inhomogeneous domain), then the corresponding free path for the new material must be accounted for. One method, called surface-tracking, requires photons to be stopped at voxel boundaries and intersections with surrounding voxels to be calculated, which can be computationally intensive for materials that have a large average free path. 
\par Alternatively, the $\delta$-tracking algorithm offers a solution by sampling the maximum cross-section $\sigma_{\text{max}}$ in the computational domain. This, in turn, brings down the average free path to the minimum in the domain. To account for this decrease in free path, the algorithm introduces $\delta$ interactions as an alternative to real interactions, resulting in no change to the energy or direction. The probability of a $\delta$ interaction $P_{\delta}$ is given by the following equation:

\begin{equation} \label{eq:4}
    P_{\delta} = \frac{\sigma_{\text{max}}(E) - \sigma(E, M)}{\sigma_{\text{max}}(E)},
\end{equation}

where $E$ is the energy of the photon undergoing the step. Note that when the photon lands in the material corresponding to the maximum cross-section, $\sigma(E, M) = \sigma_{\text{max}}$ and $P_{\delta} = 0$. On the contrary, if the photon landed in air and the domain's maximum cross-section corresponded to lead, then $\sigma(E, M) << \sigma_{\text{max}}$, making $P_{\sigma} \approx 1$.

\par Overall, $\delta$-tracking is significantly more computationally efficient for domains with similar cross-sections and can be shown to yield equivalent results to surface-tracking.

\par For more details on these algorithms, see \cite{vassiliev_monte_2017}.

\subsection{PHOTON INTERACTIONS}
If a $\delta$ interaction does not occur, then a real interaction is sampled. Therefore, the probability of a real interaction $P_r$ is directly related to $P_\delta$ by

\begin{equation}
    P_r = 1 - P_\delta,
\end{equation}
where $P_\delta$ is given by Equation \ref{eq:4}.
\\
\par If a real interaction occurs in material M, then the probability of interaction $i$ occurring is $P_i = \frac{\sigma_i(E, M)}{\sigma (E, M)}$ where $\sigma_i$ is the cross-section of interaction $i$. If there are $N$ possible interactions for a particular $E$ and $M$, then $\sigma (E, M)$ is calculated as so

\begin{equation}
    \sigma (E, M) = \sum_{i=1}^{N} \sigma_i(E, M).
\end{equation}


\par For x-rays, there are three possible photon interactions: photoelectric effect, coherent scattering, and incoherent scattering. For the photoelectric effect in MIDSX, the photon is terminated and all energy is deposited at the location of interaction. In general-purpose particle transport code systems, when a photoelectric interaction occurs, a photon of energy $E$ is absorbed by an electron in subshell $i$, causing the electron to leave the atom with energy $E_e = E - U_i$, where $U_i$ is the binding energy of the $i$th subshell. In addition, photons are emitted due to atomic relaxations. For photon energies in the medical imaging range (<120 keV), the energy of the released electrons does not allow for significant traversal through typical biological media. This limited traversal results in a localized dose distribution, in turn, validating the model used by MIDSX.

\par For coherent and incoherent scattering, the methodology of \cite{lund2018implementation} was adapted for use in MIDSX, neglecting Doppler energy broadening and the production of secondary particles.

% \begin{large}
%     \bf{1. Photoelectric Effect}
% \end{large}

% \par In the photoelectric effect model used in MIDSX, a rather simple approach is taken. When a photon interacts with an atom's electron, the photon is terminated and all energy is deposited at the location of interaction. In general purpose particle transport code systems, when a photoelectric interaction occurs, a photon of energy $E$ is absorbed by an electron in subshell $i$, causing the electron to leave the atom with energy $E_e = E - U_i$, where $U_i$ is the binding energy of the $i$th subshell. In addition, photons are emitted due to atomic relaxations. For photon energies in the medical imaging range (30 - 120 keV), the energy of the released electrons does not allow for significant traversal through typically used materials, such as tissue, bone, and fat. This limited traversal results in a localized dose distribution, in turn, validating the model used by MIDSX. \\

% \begin{large}
%     \bf{2. Coherent Scattering}
% \end{large}

% \par Thomson scattering is defined as an incoming photon of energy $E$ elastically scattering with a free electron at rest, resulting in a scattered photon of same energy $E$. The atomic DCS per unit solid angle $\Omega$ for the interaction can be derived with classical electrodynamics, and is given by

% \begin{equation}
%     \frac{d\sigma_T}{d\Omega} = r_e^2 \frac{1 + \cos^2(\theta)}{2},
% \end{equation}
% where $r_e^2$ is the classical electron radius. \\
% In an atom, photons scatter off bound electrons rather than the free electrons described by Thomson scattering, resulting in what is known as coherent (Rayleigh) scattering at . The DCS per unit solid angle $\Omega$ of the interaction, ignoring absorption edge effects, is given by

% \begin{equation}
%     \frac{d\sigma_{Co}}{d\Omega} = \frac{d\sigma_T}{d\Omega} F(x, Z),
% \end{equation}

% where $x$ is the momentum transfer between the photon and atom, $Z$ is the atomic number of the atom, and $F(x, Z)$ is the atomic form factor. $x$ is related to the scattering angle $\theta$ by


% \begin{equation}
%     x = ak\sqrt{1 - \cos\theta},
% \end{equation}

% where

% \begin{align}
%     a = \frac{m_e c^2}{\sqrt{2}hc} && \rm{and} && k = \frac{E}{m_e c^2},
% \end{align}

% where $m_e$ is the mass of an electron, $c$ is the speed of light, and $h$ is Planck's constant. \\

% The DCS per unit solid angle $\Omega$ can be integrated over $\phi$ to obtain the DCS per unit polar angle $\theta$:


% \begin{equation}
%     \frac{d\sigma_{Co}}{d\theta} = \pi r_e^2 \sin \theta (1 + \cos^2 \theta) F(x, Z)^2.
% \end{equation}

% \par The PDF of the polar angle $\theta$ is then given by

% \begin{equation}
%     p(\theta) d\theta = \frac{d\sigma_{Co}}{d\theta} \frac{1}{\sigma_{Co}} d\theta = \frac{\pi r_e^2}{\sigma_{Co}} \sin \theta (1 + \cos^2 \theta) F(x, Z)^2 d \theta.
% \end{equation}

% \par The PDF of $\theta$ can be transformed into a PDF of $\mu = \cos \theta$, resulting in

% \begin{equation}
%     p(\mu) = \frac{\pi r_e^2}{\sigma_{Co}} (1 + \mu^2) F(x, Z)^2.
% \end{equation}

% To then sample $\mu$ for a particular scattering event, the inversion method is used. In particular, a look up table for the CDF of $P(\mu)$ is generated for each material in the domain for a grid of $\mu$ values. The details of this algorithm are discussed in Appendix ... \\

% \begin{large}
%     \bf{3. Incoherent Scattering}
% \end{large}

% \par Incoherent (Compton) scattering is defined as an incoming photon of energy $E$ interacting with an atom's electron, resulting in a scattered photon of energy $E'$ and an released electron with energy $E_e = E - E' - U_i$, where $U_i$ is the binding energy of the interacting subshell. While coherent scattering effectively interacts with atom itself, incoherent scattering interacts with the electron. The DCS per unit solid angle $\Omega$ of the interaction was derived by Klein and Nishina in 1929, making it one of the first findings of quantum electrodynamics. The Klein-Nishina formula is given by

% \begin{equation}
%     \frac{d\sigma_{KN}}{d\Omega} = \frac{r_e^2}{2} \left(\frac{E'}{E}\right)^2 \left(\frac{E'}{E} + \frac{E}{E'} - \sin^2 \theta \right),
% \end{equation}

% \par Note that when $E' = E$, the KN DCS is equal to the Thomson DCS, showing that incoherent scattering is a generalization of coherent scattering for inelastic interactions. \\

% \par Applying conservation of energy and momentum to free electron at rest, the following equation can be derived relating the scattered photon energy $E'$ to the scattering angle $\theta$ and the incident photon energy $E$:

% \begin{equation}
%     E' = \frac{E}{1 + k(1 - \cos \theta)}.
% \end{equation}

% \par Similar to the Thomson DCS, the KN DCS assumes a free electron at rest. In an atom, the electron is bound, resulting in a modified DCS. In the case of incoherent scattering, the KN DCS is modified by the Coherent Scattering Function, making the DCS per unit solid angle $\Omega$ of the interaction

% \begin{equation}
%     \frac{d\sigma_{In}}{d\Omega} = \frac{d\sigma_{KN}}{d\Omega} S(x, Z).
% \end{equation}


% Instead of directly sampling the PDF of the DCS, $\mu$ is first sampled using the acceptance-rejection method developed by Ozmutlu, then $S(x, Z)$ is sampled once again with the acceptance-rejection method. The details of this algorithm are discussed in Appendix ... \\

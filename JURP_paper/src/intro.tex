\par Computed Tomography (CT) imaging is a critical diagnostic tool used by medical professionals to diagnose various illnesses and injuries. While the use of CT imaging is essential to provide immediate, life-saving results, ionizing radiation can damage cells and increase the risk of cancer. While the risk is small, it is cumulative, so physicians must track a patient's radiation exposure over time \onlinecite{lauer2009elements}. Typically, exposure is measured using absorbed dose in the body and air kerma at the skin layer, which both have units of grays (joule/kg), making it directly related to the energy deposited by photons and their secondary particles. Often, these values are estimated based on the properties of the x-ray source and the specific procedure being performed. However, the most accurate estimation techniques use Monte Carlo (MC) methods to simulate the propagation of photons through a computational phantom \cite{essmedphys2012}. 
\par In the MC technique, the 3D space encompassing the phantom and the radiation source is represented as a computational domain. Within this domain, individual photon interactions are stochastically simulated, accounting for each interaction event as photons navigate the phantom. Such stochastic simulation offers unparalleled precision, capturing even the most subtle nuances of radiation behavior in biological media. Additionally, the MC method can simulate various medium types, densities, and configurations, making it incredibly versatile and adaptable to various imaging tests. Furthermore, advancements in computational power and algorithms have expedited MC simulation, rendering it more accessible and feasible for routine clinical applications \cite{fernandez_bosman_validation_2021}.
\par In this paper, the newly developed, open-source MC photon transport code system MIDSX is presented and validated. While many existing MC transport code systems perform reliably in dosimetry applications \cite{fernandez_bosman_validation_2021, geant4valid2004}, many of these systems are tailored for general particle transport. The developmental focus of MIDSX on x-ray transport reduces the complexity of implementation and allows users to easily design and run simulations specifically relating to x-ray transport in the medical imaging energy range. The subsequent sections will delve into the theory of MIDSX, compare results from MIDSX to accepted benchmarks from established simulation systems, and outline future work.

% To track said exposure, we must first be able to quantify and determine the dosage of radiation a patient receives during an imaging test.
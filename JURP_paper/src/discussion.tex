\newpage
\section{Discussion}
\par Overall, MIDSX shows varied but reasonable agreement with the reference codes systems of TG-195. For Case 1, excellent agreement is observed, in effect validating the total and max cross-section data used in $\delta$ tracking, and the sampling of a discrete x-ray energy spectrum. This makes MIDSX a reliable and accurate option for primary particle measurements.

\par For the ROIs of Case 2, agreement is seen almost universally, except for the single incoherent scatter deposition energy in ROI 4 and 5 for the pencil beam source shown in Fig \ref{fig:ROIPGraph}. In particular, the results reached a max MPE of 10\% for ROI 5, indicating a potential error in the incoherent scattering energy/angular distribution sampling algorithm. This discrepancy is likely a result of an error in the rejection sampling algorithm employed by MIDSX. While this algorithm shows agreement for the full-field ROI 5, the geometry of the ROI, combined with the pencil beam, results in only narrow-angle scatters hitting the ROI. Since the scattering angle distribution of incoherent scattering at the medical imaging energy range becomes extremely steep at the scattering angle $\theta = 0^\circ$, there is likely numerical instability presenting itself in the algorithm that needs to be analyzed. In addition, the $15^\circ$ full-field tissue energy deposition measurements were larger than the reference code systems', with an MPE of 0.5\%. With the MPE increasing significantly from the $0^\circ$ to $15^\circ$ measurements, this hints at a possible geometric error with the source and/or body. However, the source's position and angular distribution, along with the domain's and tissue's dimensions very extensively verified, making the geometry of the scene less likely to the be source of the discrepancy.

\par For Case 5, almost all organ energy deposition results were lower than the reference code results, with the MPE reaching 6.3\%. An exception was seen with the thyroid, which had a uniquely larger energy deposition with an MPE of 2.9\%. One common error reported by TG-195 that could result in the observed discrepancies is the incorrect orientation of the voxelized phantom in the computational domain. On top of checking the scene's geometry and the .domain file, the orientation was verified by taking the RMSPE of the MIDSX data with respect to the results of each simulated angle reported by TG-195. As expected, the RMSPE with respect to $0^\circ$ was the minimum, further solidifying the belief that the phantom's orientation during the CT simulation was correct.

\par However, despite the orientation being verified, the deviation of MIDSX's energy deposition results for both Cases 2 and 5 raises concerns. This suggests that there may be other underlying issues in the MIDSX system that need further investigation. Potential factors could include the software's handling of scattering events, cross-section data initialization, and interpolation. It's imperative for future research to delve deeper into these aspects to pinpoint and rectify the source of the systematic errors observed in the MIDSX results.
% \par Overall, MIDSX shows relative agreement with the PENELOPE, EGSnrc, Geant4, and MCNP results provided by TG-195 for the three examined cases. For the HVL layer simulation described by Case 1, statistical agreement is seen with all code systems except for PENELOPE for the HVL and QVL 100 keV simulation. In addition, for the 30 keV simulation, statistical agreement is seen with PENELOPE, Geant4, and MCNP for HVL, along with PENELOPE and MCNP for QVL. Note that while no statistical agreement is observed for the other energies/thicknesses, all MIDSX results are within 0.32\% of the mean of the reference code systems.

% \par For Case 2, agreement is rather varied. For the full-field ROI measurements not shown in this paper due page constraints, very little statistical agreement is seen between the code systems; however, a $<3$\% mean percent error (MPE) is seen for MIDSX's results to each ROI simulation. Furthermore, for the pencil-beam ROI measurements shown in Fig \ref{fig:ROIPGraph}, statistical agreement is not readily observed, but a $<2.1$\% MPE is observed for each ROI simulation expect for the case of a single incoherent scatter. In this particular case, MIDSX's results for ROI 4 and 5 are significantly lower, with the MPE reaching 10\% for ROI 5. This discrepancy is likely a result of an error in the rejection sampling algorithm employed by MIDSX. While this algorithm shows agreement for the full-field ROI 5, the geometry of the ROI, combined with the pencil beam, results in only narrow angle scatters hitting the ROI. Since the scattering angle distribution of incoherent scattering at the medical imaging energy range contains a vertical asymptote approaching 0 at $\theta = 0^\circ$, there is likely some form of numerical instability presenting itself in the algorithm that needs to be analyzed.

% \par In the full-field tissue deposition measurements depicted in Fig \ref{fig:BDGraph}, we do not observe statistical agreement. However, for the $0^\circ$ case, the disagreement between code systems is minimal with an MPE of less than 0.1\% for MIDSX. Conversely, for the MIDSX results at $15^\circ$, the MPE reaches approximately 0.5\%. Despite extensive investigations into this pronounced discrepancy, a solution remains elusive.

% \par For Case 5, almost all of MIDSX's results are marginally lower than the mean of the reference code systems, with MPE's ranging from 1.1\% to 6.3\%. This pattern is disrupted by the thyroid, which is larger than the mean by 2.9\%. In order to quantify the cumulative error, the root mean square percent error (RMSPE) was calculated using each organ result, which resulted in the RMSPE for MIDSX being 5\%. In addition, with all other code systems typically having an MPE less than 1\%, except for MCNP which reaches an MPE of 2.2\% for the breast, there appears to a systematic error with the MIDSX code system with regard to the CT simulation. One common error reported by TG-195 is the incorrect orientation of the voxelized phantom in the computational domain. On top of checking the scenes geometry and the .comp file, the orientation was verified by taking the RMSPE of the MIDSX data with respect to the results of each simulated angle reported by TG-195. As expected, the RMSPE with respect the $0^\circ$ was the minimum, further solidifying the belief that the phantom's orientation during the CT simulation is accurate.

% \par However, despite the orientation being verified, the consistent deviation of MIDSX energy deposition results for both Case 2 and 5 raises concerns. This suggests that there may be other underlying issues or intricacies in the MIDSX system that need further investigation. Potential factors could include the software's handling of certain physics processes, voxel resolution, or computational approximations. It's imperative for future research to delve deeper into these aspects to pinpoint and rectify the source of the systematic errors observed in the MIDSX results.

